\section{Introduction.}

Euler's identity $e^{i\pi} = -1$ is popularly considered to be one of the most beautiful equations in mathematics. While we won't deny this fact (it is, after all, a splendid result), we have anecdotally perceived that most students only attribute this beauty to it's form, but fail to fathom its content.

This should come as no surprise when we take into account that most of the terms from this equation are \enquote{notationally abusive}. If a secondary student were to describe its meaning with what he knows, all he would be able to say is something along the lines of \enquote{the number $e$ multiplied by itself $\pi\cdot\sqrt{-1}$ times equals $-1$} (were we to find one who knows what $e$ is). That this is even a possibility is outrageous, and the result of multiple (reasonable) compromises done during the educational system.

Those who try to clarify the essence of this expression tend to fall into one of two very different approaches. The first one is the analytical one, which usually fills the page with shady equations and proofs to make its point. It is for sure rigorous, but often leaves the reader with more questions than they started with. The explanation feels completely ridden of the spirit of playful discovery that math entails, and almost doesn't go out of the realm of symbolic manipulation.

In contrast, other explanations are so exceedingly graphical and intuitive that the analogous reaction is something like \enquote{Is this even math?}. Whether that question is made from a cynic point of view or a wonder-like state is not something we wish to discuss, but we'd like to point out that this kind of reaction shows how unrigorous these proofs tend to be under the hood.

In this brief article we wish to discuss a midrground between these two methods. We will recognise the importance of getting an intuition for each term and build a toolbox for understanding what \textit{it} as a whole means and why and how it's used. We will however be rigorous and won't refrain from the math when it is needed. We won't include every proof, as we consider these to clutter understanding and be redundant among thousands of textbooks, although we'll talk about those we consider essential.

We will go through the concepts one by one, however, it is important for the reader to at least be familiar with the ideas of functions, derivatives, integrals and basic trigonometric functions ($\sin(x)$ and $\cos(x)$). We provide an introductory section as a refresher, but with very informal definitions.

\subsection{Unlearning \enquote{poisonous} mathematics.}

Knowledge is a mess. The act of learning often consists on getting wrong explanations about what something is, becoming accustomed to those explanations and, when grown familiar with them, pulling the blanket to reveal that they are, \textit{shockingly}, not the full story. This isn't criticism or an invitation to do things any other way. After all, you can't explain group theory directly to a five year old. To learn complex topics, the student needs to be able to compare them to simpler ones.

This is also how math has historically developed. Field axioms weren't born out of the blue in the whatever-th century, but are rather the result of centuries of thinkers who have extrapolated and generalized very different ideas originally meant to solve particular problems.

The issue arises when going from one level of abstraction to another and skimming over detail and rigor. Let's use fractional exponents as an example. In secondary school, students are told that $a^{\frac{p}{q}}$ means the same as $\sqrt[q]{a^p}$ for $\frac{p}{q} \in \mathbb{Q}$, and everyone is expected to go with it. If the teacher is any good, he might give some clues on why that's the case, but no more historical context is offered. This poses two problems:

\begin{itemize}
	\item It ignores the human aspect of math, making it seem like something granted from above. We won't question math's divinity, but to think of it as a set of rules which have always existed and been the same is a childish point of view. If we want the student to do more than child's play, we must enforce a solid understanding of not only what math has to offer, but of how it was developed.
	\item It lowers the student's guard. When accustomed to decimal exponents, most students won't even consider $2^{\sqrt{2}}$ to be a problem, brushing it off as just a number with many decimal places and ignoring the fact that $\sqrt{2}$ can't be expressed in the form $\frac{p}{q} \in \mathbb{Q}$. This kind of loss of rigor is what drives so many people insane over conflicting definitions, circular arguments and overall inconsistencies.
\end{itemize}

Coming back to $e$, think of how it was defined to you. Maybe it was just presented as a constant, $2.71828...$. Or perhaps as the solution to some kind of problem regarding growth rates. The most infuriating one might be the one that goes \enquote{$e$ is defined to be the number that satisfies that $\ln(e) = 1$} just to then say \enquote{$\ln(x)$ is defined to be the logarithm whose base is $e$: $\ln(x) = \log_e(x)$}.

In this article, we will redefine every term in $e^{i\pi}$ to have its own meaning, be compatible with what's already known and --- most important of all --- not be defined circularly.

\newpage
