\section{Prerequisites.}

The bulk of this article revolves around the integral of a function, so at the very least a grasp of what functions are is expected. Complex numbers will be used, but we dedicate a section to defining them thoroughly. The only trigonometric functions important for understanding this article are $\sin(x)$ and $\cos(x)$. None of their properties are needed. In this section we offer \textbf{very} simplified definitions of the theoretical minimums to understand what follows.

\subsection{Trigonometry.} \label{trig}

\subsubsection{The number $\pi$.}

The circle is a very interesting geometric shape with lots of properties, one of which is the relationship between its perimeter $P$ and its radius $r$. A circle's perimeter is \textit{proportional} to its radius, and that proportionality constant is $\tau = 6.2831...$. In other words, $P = \tau \cdot r$.

This is a nice and simple definition, but it is often obscured by the fact that historically we don't use $\tau$, but $\pi = \frac{\tau}{2}$, and so the previous expression becomes $P = 2 \pi r = \pi d$ (where $d = 2r$ is the circle's diameter). In truth, whether we use $\pi$ or $\tau$ is irrelevant, as we'll just have to correct for factors of $2$. During this article we'll use $\tau$, both because of its intuitive value and to force the reader to not rely on their preconceptions.

\subsubsection{Angles and Radians.}

Given two intersecting lines, we can take a radius $r$ from their intersection point and trace an arc with length $s$ that goes from one line to the other. We define the \textit{angle} formed by these two lines to be a proportion between the arc length and the chosen radius: $\alpha = \frac{s}{r}$.

We are used to measuring angles in degrees. We know that a right angle is $90^{\circ}$, that the sum of angles in a triangle is $180^{\circ}$, and that a full rotation around a circle is $360^{\circ}$. Choosing 360 rather than any other number to mean a full cycle is the result of historical happenstance, although it's true that its many factors make it a very convenient number to work with.

In scientific fields we tend to ditch this traditional measurement in favor of a more \enquote{objective} unit: the radian. The radian is the unit you get by just doing the quotient $\frac{s}{r}$. So for example, if the radius $r$ is $2$ and the arc length of the formed angle is $5$, the angle's value will be $\alpha = \frac{s}{r} = \frac{5}{2}$ radians.

Note that by simply scaling the above equation by a factor of $\frac{360}{\tau}$, we'll get its value in degrees. Most importantly, note that a full rotation is $\tau$ radians, and that for a unit circle ($r = 1$), the angle will always coincide with the arc length. It is standard to consider the radian \textit{dimensionless} and for trigonometric functions to be defined in terms of it.

\newpage

\subsubsection{$\sin$ and $\cos$}

Given any point $P$ in the unit circle centered at the origin which forms an angle $\alpha$ with the X axis, we define its coordinates to be $(\cos(\alpha), \sin(\alpha))$. Given any other radius $r$, one can just scale both components by $r$. This might not be the standard beginner-friendly definition, but it will suffice. Surely their relationship with triangles, similarity and periodicity is already somewhat understood by the reader.

\subsection{Calculus.}

Calculus is a branch of mathematics which focuses on analysing functions through two transformations: the derivative and the integral. The actual definitions of these concepts are massively more nuanced than what we are about to present. Whole textbooks can and have been written about them. However, for the purposes of this article, these will suffice and serve as a common ground.

\subsubsection{The derivative.}

Given a function $f(x)$, its \textit{derivative} $f'(a)$ at a given value $a$ is defined to be the rate of change of $f$ at that value. It can be thought of as the \enquote{slope} of the line tangent to $f$ at point $(a, f(a))$. We can take the derivative of $f$ at every $x$ value and define a new function $f'(x)$ to be its \textit{derivative}.

\subsubsection{The integral.}

Given a function $f(x)$, its \textit{integral} from $a$ to $b$ is the area enclosed between the X axis, $a$, $b$ and the curve drawn by the values of $f(x)$. It's denoted as $\int_{a}^{b} f(x) dx$, and it is a \textit{number}, not a function.

As a didactic asterisk, note that we can construct a function out of an integral just as we can from any other operator. So, while $\int_a^b f(x) dx$ is not a function, $g(t) = \int_a^t f(x) dx$ is (what varies here is the upper integration boundary), analogous to how $2^3$ is not a function, but $h(x) = 2^x$ is.

To reiterate, these are incredibly non-rigorous definitions that just serve as a theoretical minimum. The reader is expected to be somewhat familiar with this section already.

\subsubsection{The antiderivative.}

Given a function $f(x)$, its antiderivative is a function $g(x)$ whose derivative yields $f(x)$. Be aware that infinitely many functions satisfy this, as any constant terms in $g(x)$ will be nullified. The way to express this is another product of notational horror:

$$g'(x) = f(x) \iff g(x) = \int f(x) dx$$

This $\int$ symbol has nothing to do with the integral, and is just a notation that means \enquote{antiderivative}.

\subsubsection{The Fundamental Theorem of Calculus.}

This theorem states a very non-trivial relationship between derivatives and integrals, which is split in two parts:

$$F(x) = \int_a^x f(t) dt \iff F'(x) = f(x)$$

and

$$\int_a^b f(x) dx = F(b) - F(a)$$

The fact that $f$ is the antiderivative of $F$ allows the previous expression to be written as:

$$\int_a^b f(x) dx = \left[\int f(x) dx)\right]_{a}^{b}$$

Another result of notational slaughtering which will forever make students confuse the concepts of integral and antiderivative.

\newpage
