\section{Requisitos previos.}

El grueso de este artículo gira en torno a la integral de una función, así que se espera como mínimo un entendimiento de qué son las funciones. Se usarán números complejos, pero dedicamos una sección a definirlos minuciosamente. Las únicas funciones trigonométricas importantes para entender este artículo son $\sin(x)$ y $\cos(x)$. Ninguna de sus propiedades será necesaria. En esta sección ofrecemos definiciones \textbf{muy} simplificadas del mínimo teórico para entender lo que sigue.

\subsection{Trigonometría.} \label{trig}

\subsubsection{El número $\pi$.}

El círculo es una figura geométrica muy interesante con montones de propiedades, una de las cuales es la relación entre su perímetro $P$ y su radio $r$. El perímetro de cualquier círculo es \textit{proporcional} a su radio, y esa constante de proporcionalidad es $\tau = 6.2831...$. En otras palabras, $P = \tau \cdot r$.

Esta definición es agradable y sencilla, pero suele ser ensombrecida por el hecho de que históricamente no usamos $\tau$, sino $\pi = \frac{\tau}{2}$, así que la expresión anterior se convierte en $P = 2 \pi r = \pi d$ (donde $d = 2r$ es el diámetro del círculo). En realidad, es indistinto usar $\pi$ o $\tau$, pues simplemente habremos de corregir por un factor de $2$. Durante este artículo utilizaremos $\tau$ tanto por su valor intuitivo como para forzar al lector a no depender de sus preconcepciones.

\subsubsection{Ángulos y Radianes.}

Dadas dos líneas intersecantes, podemos tomar un radio $r$ desde su punto de intersección y trazar un arco con longitud $s$ que vaya de una línea hasta la otra. Definimos el \textit{ángulo} formado por estas dos líneas como la proporción entre la longitud del arco y el radio elegido: $\alpha = \frac{s}{r}$.

Solemos medir ángulos en grados. Sabemos que un ángulo recto son $90^{\circ}$, que la suma de ángulos de un triángulo es $180^{\circ}$, y que una vuelta completa al círculo son $360^{\circ}$. Escoger 360 en lugar de cualquier otro número para referirse a un ciclo completo es el resultado de las ocurrencias de la historia, pero es cierto que sus numerosos divisores lo hacen un número muy conveniente con el que trabajar.

En entornos científicos se tiende a abandonar esta unidad de medida tradicional en favor de una medida más \enquote{objetiva}: el radián. El radián es la unidad que se obtiene cuando simplemente se hace el cociente $\frac{s}{r}$. Entonces por ejemplo, si el radio $r$ es $2$ y la longitud del arco que forma el ángulo es $5$, el valor del ángulo será $\alpha = \frac{s}{r} = \frac{5}{2}$ radianes.

Nótese que si escalamos esta ecuación por un factor de $\frac{360}{\tau}$, obtendremos su valor en grados. Más importantemente, nótese que una revolución completa son $\tau$ radianes, y que para un círculo unidad ($r = 1$), el valor del ángulo siempre coincidirá con la longitud del arco formado. Es estándar considerar que el radián es \textit{adimensional} y que las funciones trigonométricas se definen en base a este.

\newpage

\subsubsection{$\sin$ y $\cos$}

Dado cualquier punto $P$ sobre el círculo unidad centrado en el origen que forme un ángulo $\alpha$ con el eje X, definimos sus coordenadas como $(\cos(\alpha), \sin(\alpha))$. Dado cualquier otro radio $r$, podemos simplemente escalar ambas componentes por $r$. Puede que esto no sea una definición estándar para principiantes, pero será suficiente. Estamos seguros de que su relación con triángulos, semejanza y periodicidad ya es en cierta medida entendida por el lector.

\subsection{Cálculo.}

El cálculo es una rama de las matemáticas que se centra en analizar funciones mediante dos transformaciones: la derivada y la integral. Las definiciones de estos conceptos tienen muchísimos más matices que lo que vamos a presentar. Libros de texto completos pueden y han sido escritos sobre estas. Sin embargo, para los fines de este artículo, estas servirán como punto común.

\subsubsection{La derivada.}

Dada una función $f(x)$, su \textit{derivada} $f'(a)$ en un valor dado $a$ se define como la tasa de cambio de $f$ en ese valor. Se puede pensar como la \enquote{pendiente} de la recta tangente a $f$ en el punto $(a, f(a))$. Podemos tomar la derivada de $f$ en todos los valores de $x$ y definir una nueva función $f'(x)$ como su \textit{derivada}.

\subsubsection{La integral.}

Dada una función $f(x)$, su \textit{integral} de $a$ hasta $b$ es el área encerrada entre el eje X, $a$, $b$ y la curva dibujada por los valores de $f(x)$. Se denota $\int_{a}^{b} f(x) dx$ y es un \textit{número}, no una función.

Como un paréntesis didáctico, nótese que podemos construir una función a partir de una integral al igual que podemos hacerlo a partir de cualquier otro operador. Así que mientras que $\int_a^b f(x) dx$ no es una función, $g(t) = \int_a^t f(x) dx$ sí que lo es (lo que varía es el límite superior de integración), análogo a cómo $2^3$ no es una función, pero $h(x) = 2^x$ sí.

Para reiterar, estas son definiciones increíblemente poco rigurosas que solo sirven como mínimo teórico. Se espera que el lector esté ya algo familiarizado con esta sección.

\subsubsection{La antiderivada.}

Dada una función $f(x)$, su antiderivada es una función $g(x)$ cuya derivada da $f(x)$. Infinitas funciones cumplen esto, pues cualquier término constante en $g(x)$ se anulará. La forma de expresar esto es otro producto horrendo de la notación:

$$g'(x) = f(x) \iff g(x) = \int f(x) dx$$

Este símbolo $\int$ no tiene nada que ver con la integral, solo es una notación que significa \enquote{antiderivada}.

\subsubsection{El Teorema Fundamental del Cálculo.}

Este teorema enuncia una relación para nada trivial entre derivadas e integrales, que se divide en dos partes:

$$F(x) = \int_a^x f(t) dt \iff F'(x) = f(x)$$

y

$$\int_a^b f(x) dx = F(b) - F(a)$$

El hecho de que $f$ sea la antiderivada de $F$ permite reescribir la expresión anterior como:

$$\int_a^b f(x) dx = \left[\int f(x) dx)\right]_{a}^{b}$$

Otro resulto de masacre notacional que por siempre hará que los estudiantes confundan los conceptos de integral y antiderivada.

\newpage
