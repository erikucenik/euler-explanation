\section{Números complejos.}

Los números complejos son otro tema que induce a la locura en aquéllos que no están preparados. Al estudiante se le dice durante toda su vida que \enquote{los números negativos no tienen raíces cuadradas} solo para mandar todo al garete un día y decir que mágicamente sí tienen. No solo eso, la explicación es también bastante pobre. La definición habitual suele ser \enquote{sea $i = \sqrt{-1}$}, lo cual visto desde lejos parece una forma bruta y vaga de resolver el problema. Por supuesto, los números complejos también llegaron a ser lo que son por un proceso histórico, pero preferimos ofrecer directamente el conjunto ya generalizado y bien definido.

Definimos los números complejos $\mathbb{C}$ como un álgebra sobre el espacio vectorial $\mathbb{R}^2$, siendo sus dos operaciones:

\begin{itemize}
	\item Suma: $(a, b) + (c, d) = (a + c, b + d)$.
	\item Producto: $(a, b) \cdot (c, d) = (ac - bd, ad + bc)$.
\end{itemize}

A partir de ahora, el número complejo $(a, b)$ será escrito como $a + bi$, y diremos que $a$ es su parte \textit{real} y $b$ su parte \textit{imaginaria}. Nótese que $i^2 = (0, 1) \cdot (0, 1) = (0 \cdot 0 - 1 \cdot 1, 0 \cdot 1 + 1 \cdot 0) = (-1, 0) = -1$. Debido a esto y, nuevamente, mediante algo de abuso notacional, $i = \sqrt{-1}$.

\subsection{Coordenadas polares.}

Si en lugar de tener las componentes de un número complejo $z$ se nos diese su longitud $|z|$ (también llamada su \textit{módulo}) y el ángulo $\alpha$ que forma su vector con el eje real (también llamado su \textit{argumento}), podríamos derivar sus coordenadas usando trigonometría básica (ver Sección \ref{trig}) y obtener que

$$z = |z| \cdot (\cos(\alpha) + i\cdot \sin(\alpha))$$

Estamos ahora preparados para demostrar la fórmula de Euler.

\newpage
