\section{Complex numbers.}

Complex numbers are also an insanity-inducing topic for those unprepared. The student gets told during their whole life that \enquote{negative numbers don't have square roots} just for one day to throw everything out the window and say they magically do. Not only that, the explanation is rather poor too. The usual definition goes \enquote{let $i = \sqrt{-1}$}, which, when seen from afar, seems like a lazy brutish way of solving the problem. Of course, complex numbers also came to existance through a historical process, but we prefer to offer the already-generalized and well defined set directly.

We define the complex numbers $\mathbb{C}$ as an algebra over the $\mathbb{R}^2$ vector space, with its two operations being:

\begin{itemize}
	\item Addition: $(a, b) + (c, d) = (a + c, b + d)$.
	\item Product: $(a, b) \cdot (c, d) = (ac - bd, ad + bc)$.
\end{itemize}

From now on, the complex number $(a, b)$ will be written as $a + bi$, and we will call $a$ its \textit{real} part and $b$ its \textit{imaginary} part. Note that $i^2 = (0, 1) \cdot (0, 1) = (0 \cdot 0 - 1 \cdot 1, 0 \cdot 1 + 1 \cdot 0) = (-1, 0) = -1$. Because of this fact and, again, through some notational abuse, $i = \sqrt{-1}$.

\subsection{Polar coordinates.}

If instead of having the components of a complex number $z$ we were given its length $|z|$ (also called its \textit{modulus}) and the angle $\alpha$ its vector forms with the real axis (also called its \textit{argument}), we could derive its coordinates using basic trigonometry (see Section \ref{trig}) and get that

$$z = |z| \cdot (\cos(\alpha) + i\cdot \sin(\alpha))$$

We are now ready to prove Euler's formula.

\newpage
