\section{Euler's formula.}

We are about to prove that $\exp(ix) = \cos(x) + i\sin(x)$ in two different ways.

\subsection{Proof by differential equations.}

First we establish $f$ and $g$ to be:

\begin{equation}
\begin{cases}
	f(x) = \cos(x) + i\sin(x) \\
	g(x) = \exp(ix)
\end{cases}
\end{equation}

We can see that evaluating at $x = 0$, we get the same result

\begin{equation}
\begin{cases}
	f(0) = \cos(0) + i\sin(0) = 1 \\
	g(0) = \exp(0\cdot i) = 1
\end{cases}
\end{equation}

Of course, proving it for one value is far from proving it for every value. However, note that

\begin{equation}
\begin{cases}
	f'(x) = -\sin(x) + i\cos(x) = i(\cos(x) + i\sin(x)) = if(x) \\
	g'(x) = i\exp(ix) = ig(x)
\end{cases}
\end{equation}

Two functions which satisfy the same differential equation for the same initial conditions are bound to be equal, hence

$$\exp(ix) = \cos(x) + i\sin(x)$$

\subsection{Proof by Taylor Series.}

We can write the Taylor expansion of each term:

$$\cos(x)  =  \sum\limits_{n=0}^{\infty} \frac{D^n(\cos(x))_{x=0}}{n!} x^n = 1 - \frac{x^2}{2!} + \frac{x^4}{4!} - \frac{x^6}{6!} + \cdots$$
$$i\sin(x) = i\sum\limits_{n=0}^{\infty} \frac{D^n(\sin(x))_{x=0}}{n!} x^n = ix - \frac{ix^3}{3!} + \frac{ix^5}{5!} - \frac{ix^7}{7!} + \cdots$$
$$\exp(ix) =  \sum\limits_{n=0}^{\infty} \frac{(ix)^n}{n!} = 1 + ix - \frac{x^2}{2!} - \frac{ix^3}{3!} + \frac{x^4}{4!} + \frac{ix^5}{5!} - \frac{x^6}{6!} - \frac{ix^7}{7!} + \cdots$$

It is immediately obvious that

$$\exp(ix) = \cos(x) + i\sin(x)$$

\subsection{Complex polar notation.}

Thanks to this we have a very convenient and elegant way to express complex numbers. Going back to polar coordinates, where a complex number $z$ is represented by its length $|z|$ and its angle $\alpha$, we had that

$$z = |z| \cdot (\cos(\alpha) + i\sin(\alpha))$$

With the acquired knowledge, it is clear that we can express this as

$$z = |z| \cdot \exp(i\alpha)$$

Or, if one prefers it,

$$z = |z| \cdot e^{i\alpha}$$

Et voilá

$$1\cdot e^{i\pi} = \cos(\pi) + i\sin(\pi) = -1 + 0i = -1$$
$$e^{i\pi} = -1$$

\newpage
