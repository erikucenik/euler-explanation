\section{La fórmula de Euler.}

Estamos a punto de probar que $\exp(ix) = \cos(x) + i\sin(x)$ de dos maneras distintas.

\subsection{Demostración por ecuaciones diferenciales.}

Primero establecemos que $f$ y $g$ son:

\begin{equation}
\begin{cases}
	f(x) = \cos(x) + i\sin(x) \\
	g(x) = \exp(ix)
\end{cases}
\end{equation}

Podemos ver que cuando evaluamos en $x = 0$, obtenemos el mismo resultado

\begin{equation}
\begin{cases}
	f(0) = \cos(0) + i\sin(0) = 1 \\
	g(0) = \exp(0\cdot i) = 1
\end{cases}
\end{equation}

Por supuesto, demostrarlo para un valor está lejos de demostrarlo para todos los valores, sin embargo, nótese que

\begin{equation}
\begin{cases}
	f'(x) = -\sin(x) + i\cos(x) = i(\cos(x) + i\sin(x)) = if(x) \\
	g'(x) = i\exp(ix) = ig(x)
\end{cases}
\end{equation}

Dos funciones que satisfacen la misma ecuación diferencial para la misma condicion inicial están destinadas a ser iguales, así que

$$\exp(ix) = \cos(x) + i\sin(x)$$

\subsection{Demostración por serie de Taylor.}

Podemos escribir la expansión de Taylor de cada término:

$$\cos(x)  =  \sum\limits_{n=0}^{\infty} \frac{D^n(\cos(x))_{x=0}}{n!} x^n = 1 - \frac{x^2}{2!} + \frac{x^4}{4!} - \frac{x^6}{6!} + \cdots$$
$$i\sin(x) = i\sum\limits_{n=0}^{\infty} \frac{D^n(\sin(x))_{x=0}}{n!} x^n = ix - \frac{ix^3}{3!} + \frac{ix^5}{5!} - \frac{ix^7}{7!} + \cdots$$
$$\exp(ix) =  \sum\limits_{n=0}^{\infty} \frac{(ix)^n}{n!} = 1 + ix - \frac{x^2}{2!} - \frac{ix^3}{3!} + \frac{x^4}{4!} + \frac{ix^5}{5!} - \frac{x^6}{6!} - \frac{ix^7}{7!} + \cdots$$

Es inmediatamente obvio que

$$\exp(ix) = \cos(x) + i\sin(x)$$

\subsection{Notación polar compleja.}

Gracias a esto tenemos una forma muy conveniente y elegante de expresar números complejos. Regresando a las coordenadas polares, cuando un número complejo $z$ era representado por su longitud $|z|$ y su ángulo $\alpha$, teníamos que

$$z = |z| \cdot (\cos(\alpha) + i\sin(\alpha))$$

Con el conocimiento adquirido, está claro que podemos expresar esto como

$$z = |z| \cdot \exp(i\alpha)$$

O, si se prefiere,

$$z = |z| \cdot e^{i\alpha}$$

Et voilá

$$1\cdot e^{i\pi} = \cos(\pi) + i\sin(\pi) = -1 + 0i = -1$$
$$e^{i\pi} = -1$$

\newpage
