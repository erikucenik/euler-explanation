\documentclass[a4paper, titlepage]{article}
\title{Una demostración sin sinsentidos de $e^{i\pi} = -1$.}
\author{Erik López de la Fuente}
\date{\today}

\usepackage[table]{xcolor}
\usepackage[utf8]{inputenc}
\usepackage[spanish]{babel}
\usepackage[autostyle]{csquotes}
\usepackage{amsmath}
\usepackage{caption}
\usepackage{amsfonts}
\usepackage{enumerate}
\usepackage{graphicx}
\usepackage{float}
\usepackage{makecell}
\usepackage{hyperref}
\usepackage{listings}
\usepackage{lscape}
\usepackage{accents}

\begin{document}

\maketitle

\begin{abstract}
	Una explicación completa sobre la identidad de Euler. Se asume la posesión de cierto conocimiento sobre funciones y trigonometría y cálculo muy básicos, sin embargo, se proporciona un breve (aunque no riguroso) resumen.
\end{abstract}

\tableofcontents
\newpage

\section{Introduction.}

Euler's identity $e^{i\pi} = -1$ is popularly considered to be one of the most beautiful equations in mathematics. While we won't deny this fact (it is, after all, a splendid result), we have anecdotally perceived that most students only attribute this beauty to it's form, but fail to fathom its content.

This should come as no surprise when we take into account that most of the terms from this equation are \enquote{notationally abusive}. If a secondary student were to describe its meaning with what he knows, all he would be able to say is something along the lines of \enquote{the number $e$ multiplied by itself $\pi\cdot\sqrt{-1}$ times equals $-1$} (were we to find one who knows what $e$ is). That this is even a possibility is outrageous, and the result of multiple (reasonable) compromises done during the educational system.

Those who try to clarify the essence of this expression tend to fall into one of two very different approaches. The first one is the analytical one, which usually fills the page with shady equations and proofs to make its point. It is for sure rigorous, but often leaves the reader with more questions than they started with. The explanation feels completely ridden of the spirit of playful discovery that math entails, and almost doesn't go out of the realm of symbolic manipulation.

In contrast, other explanations are so exceedingly graphical and intuitive that the analogous reaction is something like \enquote{Is this even math?}. Whether that question is made from a cynic point of view or a wonder-like state is not something we wish to discuss, but we'd like to point out that this kind of reaction shows how unrigorous these proofs tend to be under the hood.

In this brief article we wish to discuss a midrground between these two methods. We will recognise the importance of getting an intuition for each term and build a toolbox for understanding what \textit{it} as a whole means and why and how it's used. We will however be rigorous and won't refrain from the math when it is needed. We won't include every proof, as we consider these to clutter understanding and be redundant among thousands of textbooks, although we'll talk about those we consider essential.

We will go through the concepts one by one, however, it is important for the reader to at least be familiar with the ideas of functions, derivatives, integrals and basic trigonometric functions ($\sin(x)$ and $\cos(x)$). We provide an introductory section as a refresher, but with very informal definitions.

\subsection{Unlearning \enquote{poisonous} mathematics.}

Knowledge is a mess. The act of learning often consists on getting wrong explanations about what something is, becoming accustomed to those explanations and, when grown familiar with them, pulling the blanket to reveal that they are, \textit{shockingly}, not the full story. This isn't criticism or an invitation to do things any other way. After all, you can't explain group theory directly to a five year old. To learn complex topics, the student needs to be able to compare them to simpler ones.

This is also how math has historically developed. Field axioms weren't born out of the blue in the whatever-th century, but are rather the result of centuries of thinkers who have extrapolated and generalized very different ideas originally meant to solve particular problems.

The issue arises when going from one level of abstraction to another and skimming over detail and rigor. Let's use fractional exponents as an example. In secondary school, students are told that $a^{\frac{p}{q}}$ means the same as $\sqrt[q]{a^p}$ for $\frac{p}{q} \in \mathbb{Q}$, and everyone is expected to go with it. If the teacher is any good, he might give some clues on why that's the case, but no more historical context is offered. This poses two problems:

\begin{itemize}
	\item It ignores the human aspect of math, making it seem like something granted from above. We won't question math's divinity, but to think of it as a set of rules which have always existed and been the same is a childish point of view. If we want the student to do more than child's play, we must enforce a solid understanding of not only what math has to offer, but of how it was developed.
	\item It lowers the student's guard. When accustomed to decimal exponents, most students won't even consider $2^{\sqrt{2}}$ to be a problem, brushing it off as just a number with many decimal places and ignoring the fact that $\sqrt{2}$ can't be expressed in the form $\frac{p}{q} \in \mathbb{Q}$. This kind of loss of rigor is what drives so many people insane over conflicting definitions, circular arguments and overall inconsistencies.
\end{itemize}

Coming back to $e$, think of how it was defined to you. Maybe it was just presented as a constant, $2.71828...$. Or perhaps as the solution to some kind of problem regarding growth rates. The most infuriating one might be the one that goes \enquote{$e$ is defined to be the number that satisfies that $\ln(e) = 1$} just to then say \enquote{$\ln(x)$ is defined to be the logarithm whose base is $e$: $\ln(x) = \log_e(x)$}.

In this article, we will redefine every term in $e^{i\pi}$ to have its own meaning, be compatible with what's already known and --- most important of all --- not be defined circularly.

\newpage

\section{Requisitos previos.}

El grueso de este artículo gira en torno a la integral de una función, así que se espera como mínimo un entendimiento de qué son las funciones. Se usarán números complejos, pero dedicamos una sección a definirlos minuciosamente. Las únicas funciones trigonométricas importantes para entender este artículo son $\sin(x)$ y $\cos(x)$. Ninguna de sus propiedades será necesaria. En esta sección ofrecemos definiciones \textbf{muy} simplificadas del mínimo teórico para entender lo que sigue.

\subsection{Trigonometría.} \label{trig}

\subsubsection{El número $\pi$.}

El círculo es una figura geométrica muy interesante con montones de propiedades, una de las cuales es la relación entre su perímetro $P$ y su radio $r$. El perímetro de cualquier círculo es \textit{proporcional} a su radio, y esa constante de proporcionalidad es $\tau = 6.2831...$. En otras palabras, $P = \tau \cdot r$.

Esta definición es agradable y sencilla, pero suele ser ensombrecida por el hecho de que históricamente no usamos $\tau$, sino $\pi = \frac{\tau}{2}$, así que la expresión anterior se convierte en $P = 2 \pi r = \pi d$ (donde $d = 2r$ es el diámetro del círculo). En realidad, es indistinto usar $\pi$ o $\tau$, pues simplemente habremos de corregir por un factor de $2$. Durante este artículo utilizaremos $\tau$ tanto por su valor intuitivo como para forzar al lector a no depender de sus preconcepciones.

\subsubsection{Ángulos y Radianes.}

Dadas dos líneas intersecantes, podemos tomar un radio $r$ desde su punto de intersección y trazar un arco con longitud $s$ que vaya de una línea hasta la otra. Definimos el \textit{ángulo} formado por estas dos líneas como la proporción entre la longitud del arco y el radio elegido: $\alpha = \frac{s}{r}$.

Solemos medir ángulos en grados. Sabemos que un ángulo recto son $90^{\circ}$, que la suma de ángulos de un triángulo es $180^{\circ}$, y que una vuelta completa al círculo son $360^{\circ}$. Escoger 360 en lugar de cualquier otro número para referirse a un ciclo completo es el resultado de las ocurrencias de la historia, pero es cierto que sus numerosos divisores lo hacen un número muy conveniente con el que trabajar.

En entornos científicos se tiende a abandonar esta unidad de medida tradicional en favor de una medida más \enquote{objetiva}: el radián. El radián es la unidad que se obtiene cuando simplemente se hace el cociente $\frac{s}{r}$. Entonces por ejemplo, si el radio $r$ es $2$ y la longitud del arco que forma el ángulo es $5$, el valor del ángulo será $\alpha = \frac{s}{r} = \frac{5}{2}$ radianes.

Nótese que si escalamos esta ecuación por un factor de $\frac{360}{\tau}$, obtendremos su valor en grados. Más importantemente, nótese que una revolución completa son $\tau$ radianes, y que para un círculo unidad ($r = 1$), el valor del ángulo siempre coincidirá con la longitud del arco formado. Es estándar considerar que el radián es \textit{adimensional} y que las funciones trigonométricas se definen en base a este.

\newpage

\subsubsection{$\sin$ y $\cos$}

Dado cualquier punto $P$ sobre el círculo unidad centrado en el origen que forme un ángulo $\alpha$ con el eje X, definimos sus coordenadas como $(\cos(\alpha), \sin(\alpha))$. Dado cualquier otro radio $r$, podemos simplemente escalar ambas componentes por $r$. Puede que esto no sea una definición estándar para principiantes, pero será suficiente. Estamos seguros de que su relación con triángulos, semejanza y periodicidad ya es en cierta medida entendida por el lector.

\subsection{Cálculo.}

El cálculo es una rama de las matemáticas que se centra en analizar funciones mediante dos transformaciones: la derivada y la integral. Las definiciones de estos conceptos tienen muchísimos más matices que lo que vamos a presentar. Libros de texto completos pueden y han sido escritos sobre estas. Sin embargo, para los fines de este artículo, estas servirán como punto común.

\subsubsection{La derivada.}

Dada una función $f(x)$, su \textit{derivada} $f'(a)$ en un valor dado $a$ se define como la tasa de cambio de $f$ en ese valor. Se puede pensar como la \enquote{pendiente} de la recta tangente a $f$ en el punto $(a, f(a))$. Podemos tomar la derivada de $f$ en todos los valores de $x$ y definir una nueva función $f'(x)$ como su \textit{derivada}.

\subsubsection{La integral.}

Dada una función $f(x)$, su \textit{integral} de $a$ hasta $b$ es el área encerrada entre el eje X, $a$, $b$ y la curva dibujada por los valores de $f(x)$. Se denota $\int_{a}^{b} f(x) dx$ y es un \textit{número}, no una función.

Como un paréntesis didáctico, nótese que podemos construir una función a partir de una integral al igual que podemos hacerlo a partir de cualquier otro operador. Así que mientras que $\int_a^b f(x) dx$ no es una función, $g(t) = \int_a^t f(x) dx$ sí que lo es (lo que varía es el límite superior de integración), análogo a cómo $2^3$ no es una función, pero $h(x) = 2^x$ sí.

Para reiterar, estas son definiciones increíblemente poco rigurosas que solo sirven como mínimo teórico. Se espera que el lector esté ya algo familiarizado con esta sección.

\subsubsection{La antiderivada.}

Dada una función $f(x)$, su antiderivada es una función $g(x)$ cuya derivada da $f(x)$. Infinitas funciones cumplen esto, pues cualquier término constante en $g(x)$ se anulará. La forma de expresar esto es otro producto horrendo de la notación:

$$g'(x) = f(x) \iff g(x) = \int f(x) dx$$

Este símbolo $\int$ no tiene nada que ver con la integral, solo es una notación que significa \enquote{antiderivada}.

\subsubsection{El Teorema Fundamental del Cálculo.}

Este teorema enuncia una relación para nada trivial entre derivadas e integrales, que se divide en dos partes:

$$F(x) = \int_a^x f(t) dt \iff F'(x) = f(x)$$

y

$$\int_a^b f(x) dx = F(b) - F(a)$$

El hecho de que $f$ sea la antiderivada de $F$ permite reescribir la expresión anterior como:

$$\int_a^b f(x) dx = \left[\int f(x) dx)\right]_{a}^{b}$$

Otro resulto de masacre notacional que por siempre hará que los estudiantes confundan los conceptos de integral y antiderivada.

\newpage

\section{Exponentials and $e$.}

As we implied before, $e^x$ is a horrendous notation which should be exiled from beginner-friendly explanations. Hence, we shall make a compromise to forget its existance until qualified to use it. We ask the reader to discard everything they know about exponentials up until secondary education.

Gathering what few understanding seems self-consistent, $a^q$ only makes sense when $a \in \mathbb{R}$ and $q \in \mathbb{Q}$ (one can question the nature of real numbers and what it means to multiply two together, but that lies far from the scope of this article). When $q \in \mathbb{N}$, exponentiation becomes the familiar \enquote{multiply $a$ times itself $q$ times}. However it is traditionally introduced that, because $a^b a^c = a^{b + c}$ and because $\sqrt[2]{a} \cdot \sqrt[2]{a} = a$, it is reasonable to say that $a^{\frac{1}{2}}\cdot a^{\frac{1}{2}} = a^{\frac{1}{2} + \frac{1}{2}} = a$, and hence that $a^{\frac{1}{2}} = \sqrt[2]{a}$. Extrapolating this and using similar properties, we generally say that $a^{\frac{b}{c}} = \sqrt[c]{a^b}$.

Note that at no point have we mentioned real numbers as exponents, and for good reason: they don’t make sense at all. How could one define $e^x$ to be a continuous function if only rational numbers are allowed?

\subsection{The area under $\frac{1}{t}$.}

Take the function $f(t) = \frac{1}{t}$ and let $A(x) = \int_1^x f(t) dt$ (Figure \ref{graph}). One can imagine this as a plane with axes $t$ and $y$ in which a function $f(t)$ is drawn. We establish a fixed point at $t = 1$ and draw a vertical line from the $t$ axis up until its intersection with $f(t)$. To its right, we have another vertical segment at $t = x$ which is also contained between the $t$ axis and the function’s graph. One can imagine $x$ and its line as moving freely through the $t$ axis, widening and shortening the \enquote{window} between the two vertical segments. With this notion in mind, the function $A(x)$ describes the value of the area of that window. Of course, this is just an integral.

\begin{figure}[H]
	\centering
	\includegraphics[width=\linewidth]{media/lnx.png}
	\caption{Graphical representation of the integral from $1$ to $x$ of the function $f(t) = \frac{1}{t}$. The variable $x$ can be thought of as a \enquote{slider} which changes how wide the integration window is. An interactive Desmos graph is available at \url{https://www.desmos.com/calculator/drhxhrnnr1}.}
	\label{graph}
\end{figure}

Note that what \enquote{varies} in $A(x)$ is $x$, in other words, that \enquote{slider} which changes how wide our window is. $t$ doesn't vary with $x$, it just defines the function. To clarify this, one might consider taking specific values of $x$. For example, $A(3) = \int_{1}^{3} f(t) dt$ means \enquote{the area encapsulated under the curve drawn by $\frac{1}{t}$ between $t = 1$ and $t = 3$}. This yields a real number with value $1.0986...$.

We \textbf{define} that area function to be the \textit{natural logarithm of $x$} (a name which might as well be meaningless for now), written as $\ln(x) := A(x) = \int_{1}^{x} \frac{1}{t} dt$. Because of the Fundamental Theorem of Calculus, $A'(x) = (\ln(x))' = \frac{1}{x}$ (it can also be proved using the Sandwich theorem). We also define $e$ to be the number that satisfies that $\ln(e) = 1$. 

\newpage

\subsection{Properties of the $\ln(x)$ function.}

Just by looking at the graph it is immediately obvious that for the $\ln(x)$ function to make sense, $x$ must be greater than $0$. We can analyse values between $0$ and $1$ by inverting the integration boundaries and the sign of the result. It's easy to see then that $\ln(x)$ is always an increasing function, and that for values between 0 and 1, its value will be negative. Furthermore, at $x = 1$, its area gets squished into nothingness, so $\ln(1) = 0$.

We can prove the following two properties:

\begin{itemize}
	\item $\ln(ax) = \ln(a) + \ln(x)$ for all $a, x \in \mathbb{R}^+$.
	\item $\ln(x^{\frac{p}{q}}) = \frac{p}{q} \ln(x)$ for all $\frac{p}{q} \in \mathbb{Q}$ and $x \in \mathbb{R}^+$.
\end{itemize}

\subsubsection{Proof of $\ln(ax) = \ln(a) + \ln(x)$.}

We start from taking the derivative of $\ln(ax)$:

$$(\ln(ax))' = \frac{1}{ax} \cdot a = \frac{1}{x} = (\ln(x))'$$

This implies that

$$(\ln(ax) - \ln(x))' = 0$$

meaning

$$\ln(ax) - \ln(x) = c \textrm{ (constant)}$$

To obtain the value of $c$, we can evaluate these functions at $x = 1$.

$$\ln(a) - \ln(1) = c \iff c = \ln(a)$$

That is to say

$$\ln(ax) = \ln(a) + \ln(x)$$

This method also works for proving that $\ln(\frac{x}{a}) = \ln(x) - \ln(a)$. Note that this works for \textit{every positive real number}. 

\subsubsection{Proof of $\ln(x^{\frac{p}{q}}) = \frac{p}{q} \ln(x)$}

Starting again from its derivative,

$$\ln(x^{\frac{p}{q}})' = \frac{1}{x^{\frac{p}{q}}} \frac{p}{q} x^{\frac{p}{q} - 1} = \frac{p}{q} \frac{1}{x} = \frac{p}{q} (\ln(x))'$$

Again, we substract the first and the last derivatives to get zero, meaning that $\ln(x^{\frac{p}{q}}) - \ln(x) = c$ with constant $c$. In this case, the constant turns out to be $c = 0$, leaving

$$\ln(x^{\frac{p}{q}}) = \frac{p}{q} \ln(x)$$

This is valid if $x > 0$ and if $\frac{p}{q} \in \mathbb{Q}$. Still, no real exponents to be found anywhere.

\subsection{The exponential function.}

By these properties, we can confidently say that:

$$\ln(e^2) = 2\ln(e) = 2$$
$$\ln(e^3) = 3\ln(e) = 3$$
$$\ln(e^4) = 4\ln(e) = 4$$
$$\ln(e^{q}) = q, \forall q \in \mathbb{Q}$$

Now we define the \textit{exponential function} to be the inverse of the natural logarithm function: $\exp(x) := \ln^{-1}(x)$. Think of the term \textit{exponential} as a name instead of an adjective. By this definition and by the previous results:

$$\exp(2) = \ln^{-1}(2) = e^2$$
$$\exp(3) = \ln^{-1}(3) = e^3$$
$$\exp(4) = \ln^{-1}(4) = e^4$$

It can be proved that the $\exp(x)$ function shares a lot of similarities with raising $e$ to a power in this regard. The usual properties of exponents transfer into this function.

$$\exp(a + b) = \exp(a\cdot b)$$
$$\exp(a \cdot b) = \exp(a)^b$$
$$\exp\left(\frac{1}{2}\right) = \sqrt{e}$$
$$\exp(-1) = e^{-1}$$

Because of this bizarre correlation, it is almost universal to write $\exp(x)$ as $e^x$. However, $\ln: (0, \infty) \to \mathbb{R}$, and hence $\exp: \mathbb{R} \to (0, \infty)$. In other words, we can evaluate $\exp(x)$ at irrational $x$ values. This isn't a problem for $\exp(x)$, but to say that $e^x$ can have an irrational power implies redefining what \enquote{power} even means.

In essence, we'll define every power with real exponent in terms of $\ln$ and $\exp$. Taking advantage of some notational abuse, we say that:

$$e^x := \exp(x)$$
$$a^x := \exp(x\cdot \ln(a)) = e^{x\cdot \ln(a)}$$

where $a\in \mathbb{R}^+$ and $x\in \mathbb{R}$. Note that $-a^x \in \mathbb{R}$, whereas $(-a)^x \in \mathbb{C}$. This is our first formal definition of $e$:

$$e := \exp(1) = \ln^{-1}(1) \iff \ln(e) = \int_{1}^{e} \frac{1}{t} dt = 1$$

\subsection{Other definitions of $e$.}

As with many things in math, what you define in one occasion turns up in other unexpected places. $e$ was originally discovered in the study of interest rates over different periods of time by Bernoulli.

$$\exp(x) = \lim_{n \to \infty} \left(1 + \frac{x}{n}\right)^n$$

$e$ is characteristic of those phenomena in nature where the rate of change of some variable depends on the value of that variable. Expressed through differential equations, $\exp(x)$ is the solution to:

\begin{equation}
	\begin{cases}
		f'(x) = f(x) \\
		f(0) = 1
	\end{cases}
\end{equation}

If we tried to examine $f$ through its Taylor expansion, we would get another definition for the exponential:

$$\exp(x) = \lim_{n \to \infty} \sum\limits_{k = 0}^{n} \frac{x^k}{k!}$$

It's precisely \textit{this} definition of $e$ that allows us to plug into $\exp(x)$ whatever thing that can be raised and divided by a natural power, whether that be natural numbers, complex numbers or matrices. Notational abuse once again? Maybe, but its unmatched versatility makes it worth it.

\newpage

\section{Complex numbers.}

Complex numbers are also an insanity-inducing topic for those unprepared. The student gets told during their whole life that \enquote{negative numbers don't have square roots} just for one day to throw everything out the window and say they magically do. Not only that, the explanation is rather poor too. The usual definition goes \enquote{let $i = \sqrt{-1}$}, which, when seen from afar, seems like a lazy brutish way of solving the problem. Of course, complex numbers also came to existance through a historical process, but we prefer to offer the already-generalized and well defined set directly.

We define the complex numbers $\mathbb{C}$ as an algebra over the $\mathbb{R}^2$ vector space, with its two operations being:

\begin{itemize}
	\item Addition: $(a, b) + (c, d) = (a + c, b + d)$.
	\item Product: $(a, b) \cdot (c, d) = (ac - bd, ad + bc)$.
\end{itemize}

From now on, the complex number $(a, b)$ will be written as $a + bi$, and we will call $a$ its \textit{real} part and $b$ its \textit{imaginary} part. Note that $i^2 = (0, 1) \cdot (0, 1) = (0 \cdot 0 - 1 \cdot 1, 0 \cdot 1 + 1 \cdot 0) = (-1, 0) = -1$. Because of this fact and, again, through some notational abuse, $i = \sqrt{-1}$.

\subsection{Polar coordinates.}

If instead of having the components of a complex number $z$ we were given its length $|z|$ (also called its \textit{modulus}) and the angle $\alpha$ its vector forms with the real axis (also called its \textit{argument}), we could derive its coordinates using basic trigonometry (see Section \ref{trig}) and get that

$$z = |z| \cdot (\cos(\alpha) + i\cdot \sin(\alpha))$$

We are now ready to prove Euler's formula.

\newpage

\section{Euler's formula.}

We are about to prove that $\exp(ix) = \cos(x) + i\sin(x)$ in two different ways.

\subsection{Proof by differential equations.}

First we establish $f$ and $g$ to be:

\begin{equation}
\begin{cases}
	f(x) = \cos(x) + i\sin(x) \\
	g(x) = \exp(ix)
\end{cases}
\end{equation}

We can see that evaluating at $x = 0$, we get the same result

\begin{equation}
\begin{cases}
	f(0) = \cos(0) + i\sin(0) = 1 \\
	g(0) = \exp(0\cdot i) = 1
\end{cases}
\end{equation}

Of course, proving it for one value is far from proving it for every value. However, note that

\begin{equation}
\begin{cases}
	f'(x) = -\sin(x) + i\cos(x) = i(\cos(x) + i\sin(x)) = if(x) \\
	g'(x) = i\exp(ix) = ig(x)
\end{cases}
\end{equation}

Two functions which satisfy the same differential equation for the same initial conditions are bound to be equal, hence

$$\exp(ix) = \cos(x) + i\sin(x)$$

\subsection{Proof by Taylor Series.}

We can write the Taylor expansion of each term:

$$\cos(x)  =  \sum\limits_{n=0}^{\infty} \frac{D^n(\cos(x))_{x=0}}{n!} x^n = 1 - \frac{x^2}{2!} + \frac{x^4}{4!} - \frac{x^6}{6!} + \cdots$$
$$i\sin(x) = i\sum\limits_{n=0}^{\infty} \frac{D^n(\sin(x))_{x=0}}{n!} x^n = ix - \frac{ix^3}{3!} + \frac{ix^5}{5!} - \frac{ix^7}{7!} + \cdots$$
$$\exp(ix) =  \sum\limits_{n=0}^{\infty} \frac{(ix)^n}{n!} = 1 + ix - \frac{x^2}{2!} - \frac{ix^3}{3!} + \frac{x^4}{4!} + \frac{ix^5}{5!} - \frac{x^6}{6!} - \frac{ix^7}{7!} + \cdots$$

It is immediately obvious that

$$\exp(ix) = \cos(x) + i\sin(x)$$

\subsection{Complex polar notation.}

Thanks to this we have a very convenient and elegant way to express complex numbers. Going back to polar coordinates, where a complex number $z$ is represented by its length $|z|$ and its angle $\alpha$, we had that

$$z = |z| \cdot (\cos(\alpha) + i\sin(\alpha))$$

With the acquired knowledge, it is clear that we can express this as

$$z = |z| \cdot \exp(i\alpha)$$

Or, if one prefers it,

$$z = |z| \cdot e^{i\alpha}$$

Et voilá

$$1\cdot e^{i\pi} = \cos(\pi) + i\sin(\pi) = -1 + 0i = -1$$
$$e^{i\pi} = -1$$

\newpage

después de haber definido qué es $\ln(x)$ podemos extender $\log_e(x)$ para significar $\ln(x)$.

Por supuesto, después de haber definido y demostrado todo rigurosamente podemos escribir las cosas en cualquier orden, pero esto solo debe hacerse una vez el estudiante ha obtenido un entendimiento sólido sobre por qué estas retrocompatibilidades funcionan.

En una nota menos técnica desearíamos señalar que elegir $e^x$ para representar fenómenos exponenciales en disciplinas como la física o la estadística es meramente una decisión estética. Uno podría perfectamente expresar $2^3$ como $\pi^{\log_{\pi}(2)\cdot 3}$, es solo que al elegir $e^x$, solo las constantes relevantes --- las que de verdad poseen significado semántico en el universo --- permanecen.

\newpage

Véase el caso del decaimiento exponencial. Si se tiene un material radioactivo, se puede medir cuántos isótopos radioactivos tiene ($N_0$). Con el paso del tiempo ($t$), algunos de esos isótopos se desintegrarán, transformándose en otros elementos o isótopos y dejando una menor cantidad de isótopos radioactivos en el material ($N(t)$). Este proceso también es una exponencial, y por ende puede expresarse con la siguiente ecuación diferencial:

$$N'(t) = -\lambda N(t)$$

Una posbile solución a esta ecuación es

$$N(t) = N_0 a^{-ct}$$

Donde $c = \lambda \cdot \log_a(e)$. Esto es, sin falta, una solución válida. Su único problema, como hemos dicho, es uno estético. ¿Qué significa $c$? ¿Podría uno saber algo acerca del fenómeno descrito solo mirando a esta constante? De ninguna manera. Si en su lugar lo expresamos así:

$$N(t) = N_0 e^{-\lambda t}$$

De repente no hay \enquote{constantes basura} y, más importantemente, $\lambda$ adquiere significado semántico. En el caso de la desintegración radioactiva, $\lambda$ es llamada la \textit{constante de desintegración} o \textit{decaimiento}. Esto no solo es relevante para $\lambda$, sino para todas las magnitudes que se deriven de ella. Por ejemplo, $t_{1/2} = \frac{\ln(2)}{\lambda}$ es la \textit{semivida} de la cantidad, y describe la cantidad de tiempo necesaria para que la cantidad $N$ se reduzca a la mitad. ¿Podríamos haber expresado lo mismo arrastrando un factor de $\log_a(e)$ constantemente? Totalmente, pero entonces nuestra definición de semivida sería algo como \enquote{la cantidad de tiempo necesaria para que la cantidad $N$ se reduzca a la mitad dividida por $\log_a(e)$}. Esta definición simplemente se siente embrollada y mal, mientras que la otra parece arrastrar cierto significado objetivo, como si la constante $e$ fuese la elección \textit{natural} de base. Esto es lo que hace \textit{natural} al logaritmo natural.

En resumen, las variables exponenciales son aquéllas cuya tasa de cambio es proporcional al valor de la variable en sí, y $e$ es el número que cumple que esa constante de proporcionalidad es $1$, lo que lo hace la forma más \textit{natural} de expresar muchas ecuaciones en física.

\subsection{Números complejos como rotaciones.}

Cuando se presentan los números complejos por primera vez, una de las extrañas decisiones que destacan sobre su definición es su posición en el plano complejo. Aunque cualquier par ordenado puede ser ubicado en una gráfica, no es evidente por qué $i$ debería estar a un paso del $0$ perpendicular al eje real. Sin embargo, cuando movemos nuestra perspectiva sobre los números para pensar en ellos como \textit{acciones}, esto no solo se convierte en un resultado razonable, sino en el único posible.

Si uno está familiarizado con simples vectores en $\mathbb{R}^2$, uno ya vislumbrará cómo los números pueden ser vistos de esta manera. Multiplicar un vector por un número positivo es \textit{escalarlo}, hacerlo más grande o más pequeño preservando la dirección a la que \enquote{apunta}. De manera similar, multiplicarlo por un número negativo corresponde a hacer lo mismo pero \enquote{volteando} su dirección, o rotándolo por $180^{\circ}$.

¿Pero qué es una rotación de $180^{\circ}$ si no dos rotaciones consecutivas de $90^{\circ}$? Si multiplicar por $-1$ produce una rotación de $180^{\circ}$, aquél número que represente una rotación de $90^{\circ}$ ha de ser, en cierto sentido, la raíz cuadrada de $-1$. Esta noción de multiplicaciones (de números) transformándose en sumas (de ángulos) debería dar una pista de que las exponenciales entrarán en escena tarde o temprano.

Si uno no queda convencido con este argumento, nótese lo siguiente. Si $z = a + bi$, entonces $i\cdot z = -b + ai$. Esta transformación de coordenadas es exactamente lo mismo que una rotación antihoraria de $90^{\circ}$ ($(x, y) \to (-y, x)$). Este concepto puede extrapolarse a otros números complejos, sin embargo nuestro caso solo requiere que entendamos la acción de $i$.

\subsection{Exponenciales complejas.}

Esta es quizás la parte más difícil de visualizar, así que usaremos una analogía con la física. Puede ser útil interpretar $f(t) = \exp(it)$ como la posición de una partícula en el plano complejo en cierto punto en el tiempo, y su derivada $f'(t) = i\exp(it)$ como su velocidad. Vale la pena notar que, en cualquier posición, el vector velocidad de la partícula forma un ángulo de $90^{\circ}$ con su vector posición (pues $f'(t) = i\cdot f(t)$, esto es, una rotación antihoraria de $90^{\circ}$. Uno puede visualizar a la partícula \enquote{orbitando} en torno al origen en un movimiento circular (Figura \ref{rot}).

\begin{figure}[H]
	\centering
	\includegraphics[width=\linewidth]{media/rotation.png}
	\caption{Partícula en el plano complejo con posición $\exp(it)$ (señalada en rojo) y velocidad $i\exp(it)$ (señalada en morado). Al principio, $f(0) = 1$. Después de un cambio infinitesimal en el tiempo, la partícula se habrá movido perpendicular a su vector posición (a lo largo de su vector velocidad). Como sabemos por la física, este es el escenario para el movimiento circular. Un gráfico interactivo de Desmos está disponible en \url{https://www.desmos.com/calculator/8npnlxgjba}.}
	\label{rot}
\end{figure}

Lo que nos dice la fórmula de Euler es que el ángulo barrido por el vector posición de la partícula en un tiempo $t$ es \textit{exactamente} $t$. Entonces, tras 2 segundos, la partícula forma un ángulo de 2 radianes con el eje real; después de $\pi$ segundos, la partícula forma un ángulo de $\pi$ radianes con el eje real (lo cual coincide con el punto $-1 + 0i$), etc.

Nuevamente, la función exponencial aquí no es una decisión fundamental. Podríamos perfectamente haber elegido el número áureo $\phi$ como una base, haciendo que $f(t) = \phi^{it}$. La única diferencia es que en este caso, los ángulos barridos no coinciden con el tiempo transcurrido. Usando esta función, llegamos a la fórmula de EulErik (nombrada así por ningún motivo en particular):

$$\phi^{i\frac{\pi}{\log_{\phi}(e)}} = -1$$

\section{Conclusión.}

Para condensar todo lo visto hasta ahora en un resumen sencillo:

\begin{itemize}
	\item Los números complejos proporcionan una forma de expresar rotaciones en el plano complejo. Concretamente, $i$ significa una rotación antihoraria de $90^{\circ}$.
	\item La función exponencial ($\exp(x)$ o $e^x$) proporciona una forma de expresar variables cuya tasa de cambio es igual a su valor actual. Su característica más fundamental es el hecho de que su derivada es igual a sí misma.
	\item $e^{it}$ es el punto que resulta de rotar un punto en torno al origen por tiempo $t$, y por propiedades expuestas de $\exp(x)$, este caso particular satisface que el tiempo transcurrido es igual al ángulo barrido por el vector posición del punto. Otra forma de expresar la posición de este vector es $\cos(x) + i\sin(x)$ (la fórmula de Euler).
	\item $e^{i\pi}$ es el caso particular donde el punto aterriza en $-1$ (la identidad de Euler).
\end{itemize}


\end{document}
